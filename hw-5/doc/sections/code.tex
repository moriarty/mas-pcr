\newpage
\textbf{Question 4:}

\emph{The missionaries and cannibals problem is usually stated as follows. Three missionaries and three cannibals are on one side of a river, along with a boat that can hold one or two people. Find a way to get every- one to the other side, without ever leaving a group of missionaries in one place outnumbered by the cannibals in that place.}

\begin{enumerate}[a)]
\item \emph{Formulate the problem precisely, making only those distinctions necessary to ensure a valid solution. Draw a diagram of the complete state space.}

It is necessary to keep for each side of the river the number of monks, cannibals and boats (the boat in this case is a boolean, 0 or 1, but extended versions of this problem exist). Thus six values define a state. There are a number of actions available; however some result in invalid states, where cannibals outnumber monks. The actions can be encompassed in 3 integer values, representing the change in number of monks, cannibals or boats from one site of the river to the other (with the sign indicating direction.)

In the following diagram, the direction of the transition has been ignored for a cleaner graph.

Note: Having error with graphviz, including image and last working version of graph. State spaces accord to those in image
%% 

\item \emph{Implement and solve the problem optimally using an appropriate search algorithm. Is it a good idea to check for repeated states?}

Yes. Without checking for repeated states it is very easy to initially or finally get stuck in loops.

\item \emph{Why do you think people have a hard time solving this puzzle, given that the state space is so simple?}

Because most people take a greedy approach and do not think to bring cannibals or monks back across the river. To many this is a step away from the solution; analogous to being stuck in a local minima while performing a gradient decent to find a global minima.  

\end{enumerate}

